\section{Delimitation}\label{sec:Delimitation}

% Search and Rescue
In Section \ref{sec:Search_and_Rescue}, the origin of the problem and further details about it was explored. The reason why sound-based robotics is a useful tool for exploring the environment in a search and rescue operation, is that the environment can be visually challenging to sense, due to e.g. smoke or transparent surfaces. Due to the circumstances of search and rescue, it is essential to have a fast and efficient solution for mapping as much area in as little time as possible. Furthermore, the solution may also be able to localise people calling for help.

% Audio-Based Navigation
In Section \ref{sec:AudioBasedNavigation}, the overall challenge of mapping the environment using sound-based techniques was explored. The frequency of sound could be split into three categories, i.e. infrasound, audible, and ultrasound. The project will deal with the audible spectrum, as the available sensing hardware consist of ordinary measurement microphones operable in the audible range. These are condenser microphones with omnidirectional pickup patterns. A loudspeaker, corresponding to the range of the microphones, is used together with these. In this way, \gls{SLAM} can be combined with echolocation to obtain a proper map of the environment. The echolocation must be able to overcome some levels of noise, meaning that it can operate under certain \gls{SNR} conditions. Subsequently, a categorisation of path planners and their general architecture was introduced. From these findings, a navigation architecture that integrates both a global and local path planner will be implemented for the solution proposal. This ensures an obstacle-free and smooth path with improved capabilities for escaping dead-ends. Theory on exploration themes was afterwards presented. The three main characteristics to consider in a multi-robot scenario were given to be the central place, the size of the the environment, and the presence of obstacles. It was found that exploration schemes are categorised by the extremes \textit{Random Exploration} and \textit{Systematic Exploration}. As the topic of this project deals with unknown environments, the choice of minimal navigation solution will be a combination of these extremes. 

Based on the analysis of the robotic locomotion utilised for the solution, it could be derived that drones were the best approach for the given circumstances. However, as only the TurtleBot2 robotic platform is available, the solution will traverse on wheels. This further entails that the odometry sensors used, will be the ones already integrated on the TurtleBot2. 



%, which resulted in the two extremes \textit{Random Exploration} and \textit{Systematic Exploration}. As the topic of this project deals with unknown environments, the choice of minimal navigation will be chosen as a combination of the two extremes.

In Section \ref{sec:Swarm_Robotics}, multi-robot exploration in a constrained environment was explored. A higher number of robots are complementary to the search and rescue missions, because they increase the area that can be mapped in a given time frame. While a connection between the robots is essential, it can both occur completely locally on each robot or through a centralised hub. For the scope of this project, the latter is used for data storage and communication. Furthermore, due to limited hardware, the swarm-based part of the project will only be done in simulations rather than in real life test scenarios. 

