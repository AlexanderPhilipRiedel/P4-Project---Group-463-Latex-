\usepackage[utf8]{inputenc} % symboler, såsom æøå eller lignende
\usepackage[danish,english]{babel} % rapportens sprog

%%%% COLOR PACKAGES
\usepackage[dvipsnames]{xcolor} % For custom colors

\usepackage{tcolorbox} % Color and can makes boxes 
    \tcbset{nobeforeafter, width= 0mm, boxsep=0mm, left=2mm, right=2mm, height=4mm, colframe=black}

%%%% LINKS PACKAGES
\usepackage[hidelinks]{hyperref} % interaktive referencer!
    \hypersetup{
    breaklinks=true,
    urlcolor=black,
    colorlinks=true,
    citecolor=black,
    filecolor=black,
    linkcolor=black,
    pdftitle={P3 - Project},
    }

%%%% PAGE SETUP
\usepackage[left = 2cm, right = 2cm, bottom = 3cm, top = 2.5cm]{geometry} % Used to adjust the margins
\usepackage{chngpage} % Cover page
\usepackage{tocloft} % Table of contents
    \setcounter{secnumdepth}{3} % The higher the number the more layers of sections can be used
    \setcounter{tocdepth}{1} % Sets how many layers are shown in the table of contents
\usepackage{fancyhdr} % Allows for header and footer adjustments
    \fancyhf{} % Resets the header and footer
    \fancypagestyle{plain} % Redefines the plain pagestyle
    {
        \fancyhf{} % Resetter side hoved og fod
        \setlength{\headheight}{15pt}
        \setlength{\voffset}{5pt}
        \setlength{\textheight}{710pt}
        \renewcommand{\headrulewidth}{0pt} % Fjerner linjen øverst under sidehovedet
        \fancyfoot[R]{Page \thepage \ of \pageref{LastPage}} % Viser sidenummeret nederst til højre.
    }
\usepackage[compact]{titlesec} % Til brugerdefinerede sidehoveder og kapitel og section formatering
    \titleformat{\chapter}{\fontsize{25pt}{25pt}\bfseries\color{black}}{{\color{chapterNumColor}\thechapter}}{15pt}{} % sætter fonttypen, skriftstørrelsen og farven på kapitel
    \titleformat{\part}{\fontsize{40pt}{45pt}\bfseries\centering\color{black}}{{\color{chapterNumColor}\thepart}}{15pt}{}[\thispagestyle{empty}\addtocounter{page}{-1}] % sætter fonttypen, skriftstørrelsen og farven på parts
    \titlespacing*{\chapter}{0pt}{-50pt}{10pt} 
    \titlespacing*{\section}{0pt}{10pt}{5pt}
    \titlespacing*{\subsection}{0pt}{5pt}{0pt}
    \titlespacing*{\subsubsection}{0pt}{3pt}{-7pt}
    % Titlespacing: \titlespacing{command}{left spacing}{before spacing}{after spacing}
    \setlength{\parindent}{0pt} % Indentstørrelse ved ny sektion
    \setlength{\parskip}{3mm} % Laver mellemrum ved double enter
\usepackage[style=super, toc, nopostdot, nonumberlist]{glossaries} % Allows for a glossary
    \renewcommand{\glsnamefont}[1]{\textbf{#1}}
    \renewcommand{\glsgroupskip}{\vspace{4mm}}
\usepackage{lastpage} % Allows for the \LastPage command used in the footer
\usepackage[bottom, marginal]{footmisc} % Footnotes

\usepackage{lscape} % Landscaped page style
\usepackage{pdflscape} % Landscaped pdf pages

%%%% BIBLIOGRAPHY PACKAGES
\usepackage[style=numeric, sorting=none, maxbibnames=99, sortcites = true]{biblatex} % Allows for 
    % Bib resources 
    \addbibresource{sources/2problem_analysis.bib} 
    \addbibresource{sources/3requirements.bib}
    \addbibresource{sources/4solution_proposal.bib}
    \addbibresource{sources/5modelling.bib}
    \addbibresource{sources/6Implementation.bib}
    \addbibresource{sources/7Testing.bib}
    \addbibresource{sources/8Discussion.bib}
    

%%%% TEXT SYMBOLS
\usepackage{textcomp} % flere symboler, såsom €
\usepackage{fontawesome}
\usepackage{pifont}% http://ctan.org/pkg/pifont
\usepackage{gensymb}

%%%% FIGURE PACKAGES
\usepackage{graphicx} % Muliggør brug af figurer
\usepackage{float} % Muliggør positionering af figurer
\usepackage{array}
\usepackage{wrapfig}
\usepackage[font = small, labelfont = bf]{caption}
    \setlength{\belowcaptionskip}{-5mm} % Sets the vertical distance after a caption
\usepackage{subcaption} % Used to make two pictures below/beside eachother.
\usepackage{svg} % To use svg pictures
\usepackage{pdfpages} % To use pdf pictures
\usepackage{tikz}
    \def\checkmark{\tikz\fill[scale=0.4](0,.35) -- (.25,0) -- (1,.7) -- (.25,.15) -- cycle;}
    \newcommand{\cmark}{\ding{51}}%
    \newcommand{\xmark}{\ding{55}}%
    \newcommand{\crossmark}{\ding{53}}

%%%% TABLE PACKAGES
\usepackage{tabularx}
\usepackage{colortbl} % Table colors
\usepackage{longtable}
\usepackage{multirow}
\usepackage{multicol} % Til at lave 2 ligninger ved siden af hinanden.   
\usepackage{booktabs} % For at definere /toprule og /midrule
\usepackage{boldline}
    \newcolumntype{?}{!{\vrule width 4\arrayrulewidth}} % Laver fed vertikal linje
    \newcommand{\boldline}{\Xhline{4\arrayrulewidth}} % Laver fed horisontal linje
    
    \setlength{\tabcolsep}{1.5mm} % Det horisontale indhak for tekst i tabel
    \renewcommand{\arraystretch}{1.2} % Den vertikale afstand mellem hver række i tabel
    
    %% Custom column types
    \newcolumntype{M}[1]{m{#1}<{\Centering}} %Kolonnetype med horisontal centrering
    \newcolumntype{P}[1]{p{#1}<{\Centering}} 
\usepackage[export]{adjustbox}
\usepackage{diagbox}

%%%% MATH PACKAGES
\usepackage{mathtools} % Til at bruge matematiske udtryk eks. SUM tegn.
\usepackage{mathptmx} % Til brugerdefinerede font størrelser
\usepackage{mathrsfs}%Til at lave lagrangian "L". 
\usepackage{amsmath} %Til at skrive vektor (bmatrix). 
\usepackage{amssymb}%Scale down equations
\usepackage{leftidx} %Bruges til sub og superscripts til venstre for bogstavet. https://tex.stackexchange.com/questions/30554/superscripts-before-a-letter-in-math
\usepackage{calc}
\usepackage{stmaryrd} % Avancerede matematik-udvidelser
\usepackage{nccmath}
\usepackage{siunitx} % Flot og konsistent præsentation af tal og enheder med \si{enhed} og \SI{tal}{enhed}
\sisetup{output-decimal-marker = {,}} % Opsætning af \SI og decimalseparator
\usepackage[normalem]{ulem} %math
\usepackage{nicefrac}
    %% Easier nicefrac command
    \newcommand{\nfrac}[2]{$\nicefrac{\textrm{#1}}{\textrm{#2}}$}
\usepackage{xfrac}  

% Dot product
\makeatletter
\newcommand*\dotp{\mathpalette\dotp@{.5}}
\newcommand*\dotp@[2]{\mathbin{\vcenter{\hbox{\scalebox{#2}{$\m@th#1\bullet$}}}}}
\makeatother

%%%% LISTS PACKAGES
\usepackage{enumitem} % For lists
    \setlist[itemize]{align=left, leftmargin = 0.9cm, labelsep = -3mm, topsep = 0mm} 
    \setlist[enumerate]{itemsep = 0mm, topsep = 0mm, labelsep=3mm} 
    \setlist[description]{style=nextline, leftmargin=*, itemindent = 0mm, labelsep = 0mm, labelindent = 0mm, topsep=-2mm, font=\itshape}

%%% COLOR DEFINES
\definecolor{aaublue}{RGB}{33,26,82} % Dark blue
\definecolor{chapterNumColor}{RGB}{150, 150, 150}
\definecolor{green}{RGB}{41,226,10}
\definecolor{lightgreen}{RGB}{200,255,101}
\definecolor{yellow}{RGB}{255,203,32}
\definecolor{lightred}{RGB}{255,108,100}
\definecolor{red}{RGB}{255,0,0}

%%%% CODE PACKAGES
\usepackage{minted}
    %% Code colors
    \definecolor{codebg}{rgb}{0.95,0.95,0.95}
    
    \newfloat{Snippet}{H}{}[section] % Ikke sikker på hvad den gør
    
    \newcommand{\code}[4]{% Generel code setup struktur
        \begin{Snippet}
            \centering
            \inputminted[linenos, framesep=2mm, bgcolor=codebg,fontsize=\footnotesize,baselinestretch=1.2, breaklines]{#1}{#2}
            \caption{#3}
            \label{#4}
        \end{Snippet}
    }
    \newcommand{\cpp}[3]{% Short hand c++ snippets
        \code{cpp}{#1}{#2}{#3}
    }
    \newcommand{\matlab}[3]{% Short hand matlab snippets
        \code{matlab}{#1}{#2}{#3}
    }

%%%% MISC PACKAGES
\usepackage{lipsum} % For dummy text
\usepackage{microtype} % For bedre stretch håndtering og sådan... Generelt bare bedre microtypografi

%%%% MISC DEFINES
\DeclareFieldFormat{postnote}{p. #1} % ændrer "page" til "s." i kilder ved enkelte sider
\DeclareFieldFormat{multipostnote}{p. #1} % ændrer "pages" til "s." i kilder ved interval af sider
\DeclareFieldFormat{url}{\newline\mkbibacro{URL}\addcolon\nobreakspace\url{#1}}
\urlstyle{same}
\newcommand{\<}{\textless} % Laver mindre end
\renewcommand{\>}{\textgreater} % Laver større end




%%% DON'T KNOW %%%
\usepackage{csquotes} % bruges af babel pakken
\usepackage{bm}
\usepackage{rotating}
\usepackage{dirtytalk}
\usepackage{pgf-pie}
\usepackage{pgfplots}
    \pgfplotsset{compat=1.17}
\usepackage{ragged2e}
\usepackage{setspace}
\usepackage{makecell}